\documentclass[]{article}
\usepackage{lmodern}
\usepackage{amssymb,amsmath}
\usepackage{ifxetex,ifluatex}
\usepackage{fixltx2e} % provides \textsubscript
\ifnum 0\ifxetex 1\fi\ifluatex 1\fi=0 % if pdftex
  \usepackage[T1]{fontenc}
  \usepackage[utf8]{inputenc}
\else % if luatex or xelatex
  \ifxetex
    \usepackage{mathspec}
  \else
    \usepackage{fontspec}
  \fi
  \defaultfontfeatures{Ligatures=TeX,Scale=MatchLowercase}
\fi
% use upquote if available, for straight quotes in verbatim environments
\IfFileExists{upquote.sty}{\usepackage{upquote}}{}
% use microtype if available
\IfFileExists{microtype.sty}{%
\usepackage{microtype}
\UseMicrotypeSet[protrusion]{basicmath} % disable protrusion for tt fonts
}{}
\usepackage[margin=1in]{geometry}
\usepackage{hyperref}
\hypersetup{unicode=true,
            pdftitle={CATS\_cam\_EDA},
            pdfauthor={JHMoxley},
            pdfborder={0 0 0},
            breaklinks=true}
\urlstyle{same}  % don't use monospace font for urls
\usepackage{color}
\usepackage{fancyvrb}
\newcommand{\VerbBar}{|}
\newcommand{\VERB}{\Verb[commandchars=\\\{\}]}
\DefineVerbatimEnvironment{Highlighting}{Verbatim}{commandchars=\\\{\}}
% Add ',fontsize=\small' for more characters per line
\usepackage{framed}
\definecolor{shadecolor}{RGB}{248,248,248}
\newenvironment{Shaded}{\begin{snugshade}}{\end{snugshade}}
\newcommand{\KeywordTok}[1]{\textcolor[rgb]{0.13,0.29,0.53}{\textbf{{#1}}}}
\newcommand{\DataTypeTok}[1]{\textcolor[rgb]{0.13,0.29,0.53}{{#1}}}
\newcommand{\DecValTok}[1]{\textcolor[rgb]{0.00,0.00,0.81}{{#1}}}
\newcommand{\BaseNTok}[1]{\textcolor[rgb]{0.00,0.00,0.81}{{#1}}}
\newcommand{\FloatTok}[1]{\textcolor[rgb]{0.00,0.00,0.81}{{#1}}}
\newcommand{\ConstantTok}[1]{\textcolor[rgb]{0.00,0.00,0.00}{{#1}}}
\newcommand{\CharTok}[1]{\textcolor[rgb]{0.31,0.60,0.02}{{#1}}}
\newcommand{\SpecialCharTok}[1]{\textcolor[rgb]{0.00,0.00,0.00}{{#1}}}
\newcommand{\StringTok}[1]{\textcolor[rgb]{0.31,0.60,0.02}{{#1}}}
\newcommand{\VerbatimStringTok}[1]{\textcolor[rgb]{0.31,0.60,0.02}{{#1}}}
\newcommand{\SpecialStringTok}[1]{\textcolor[rgb]{0.31,0.60,0.02}{{#1}}}
\newcommand{\ImportTok}[1]{{#1}}
\newcommand{\CommentTok}[1]{\textcolor[rgb]{0.56,0.35,0.01}{\textit{{#1}}}}
\newcommand{\DocumentationTok}[1]{\textcolor[rgb]{0.56,0.35,0.01}{\textbf{\textit{{#1}}}}}
\newcommand{\AnnotationTok}[1]{\textcolor[rgb]{0.56,0.35,0.01}{\textbf{\textit{{#1}}}}}
\newcommand{\CommentVarTok}[1]{\textcolor[rgb]{0.56,0.35,0.01}{\textbf{\textit{{#1}}}}}
\newcommand{\OtherTok}[1]{\textcolor[rgb]{0.56,0.35,0.01}{{#1}}}
\newcommand{\FunctionTok}[1]{\textcolor[rgb]{0.00,0.00,0.00}{{#1}}}
\newcommand{\VariableTok}[1]{\textcolor[rgb]{0.00,0.00,0.00}{{#1}}}
\newcommand{\ControlFlowTok}[1]{\textcolor[rgb]{0.13,0.29,0.53}{\textbf{{#1}}}}
\newcommand{\OperatorTok}[1]{\textcolor[rgb]{0.81,0.36,0.00}{\textbf{{#1}}}}
\newcommand{\BuiltInTok}[1]{{#1}}
\newcommand{\ExtensionTok}[1]{{#1}}
\newcommand{\PreprocessorTok}[1]{\textcolor[rgb]{0.56,0.35,0.01}{\textit{{#1}}}}
\newcommand{\AttributeTok}[1]{\textcolor[rgb]{0.77,0.63,0.00}{{#1}}}
\newcommand{\RegionMarkerTok}[1]{{#1}}
\newcommand{\InformationTok}[1]{\textcolor[rgb]{0.56,0.35,0.01}{\textbf{\textit{{#1}}}}}
\newcommand{\WarningTok}[1]{\textcolor[rgb]{0.56,0.35,0.01}{\textbf{\textit{{#1}}}}}
\newcommand{\AlertTok}[1]{\textcolor[rgb]{0.94,0.16,0.16}{{#1}}}
\newcommand{\ErrorTok}[1]{\textcolor[rgb]{0.64,0.00,0.00}{\textbf{{#1}}}}
\newcommand{\NormalTok}[1]{{#1}}
\usepackage{graphicx,grffile}
\makeatletter
\def\maxwidth{\ifdim\Gin@nat@width>\linewidth\linewidth\else\Gin@nat@width\fi}
\def\maxheight{\ifdim\Gin@nat@height>\textheight\textheight\else\Gin@nat@height\fi}
\makeatother
% Scale images if necessary, so that they will not overflow the page
% margins by default, and it is still possible to overwrite the defaults
% using explicit options in \includegraphics[width, height, ...]{}
\setkeys{Gin}{width=\maxwidth,height=\maxheight,keepaspectratio}
\IfFileExists{parskip.sty}{%
\usepackage{parskip}
}{% else
\setlength{\parindent}{0pt}
\setlength{\parskip}{6pt plus 2pt minus 1pt}
}
\setlength{\emergencystretch}{3em}  % prevent overfull lines
\providecommand{\tightlist}{%
  \setlength{\itemsep}{0pt}\setlength{\parskip}{0pt}}
\setcounter{secnumdepth}{0}
% Redefines (sub)paragraphs to behave more like sections
\ifx\paragraph\undefined\else
\let\oldparagraph\paragraph
\renewcommand{\paragraph}[1]{\oldparagraph{#1}\mbox{}}
\fi
\ifx\subparagraph\undefined\else
\let\oldsubparagraph\subparagraph
\renewcommand{\subparagraph}[1]{\oldsubparagraph{#1}\mbox{}}
\fi

%%% Use protect on footnotes to avoid problems with footnotes in titles
\let\rmarkdownfootnote\footnote%
\def\footnote{\protect\rmarkdownfootnote}

%%% Change title format to be more compact
\usepackage{titling}

% Create subtitle command for use in maketitle
\newcommand{\subtitle}[1]{
  \posttitle{
    \begin{center}\large#1\end{center}
    }
}

\setlength{\droptitle}{-2em}

  \title{CATS\_cam\_EDA}
    \pretitle{\vspace{\droptitle}\centering\huge}
  \posttitle{\par}
    \author{JHMoxley}
    \preauthor{\centering\large\emph}
  \postauthor{\par}
      \predate{\centering\large\emph}
  \postdate{\par}
    \date{9/13/2018}


\begin{document}
\maketitle

This is a generalized EDA document for quickly working up biologging
deployments returned from CATS tags on White Sharks. Test built on
APT\_CC0705\_07182018, based on EDA files associated with this
deployment.

rmarkdown::render(`CATS\_preprocess\_RMrkD.Rmd',params= list( deploy.key
= ``APT\_CC0705\_07182018''))

\begin{Shaded}
\begin{Highlighting}[]
\NormalTok{params}
\end{Highlighting}
\end{Shaded}

\begin{verbatim}
## $dd
## [1] "/Volumes/Avalon"
## 
## $deploy.key
## [1] "APT_CC0704_20180803"
## 
## $utc.offset
## [1] -7
## 
## $datFreq
## [1] 50
## 
## $datFreq.desired
## [1] 1
## 
## $smoothing
## [1] 5
## 
## $SI
## [1] TRUE
## 
## $sensors
## [1] "acc" "mag"
\end{verbatim}

\subsection{Workup}\label{workup}

-needs to be able to trim data

\subsection{Smoothing, isolation \&
manipulation}\label{smoothing-isolation-manipulation}

\begin{Shaded}
\begin{Highlighting}[]
\NormalTok{df <-}\StringTok{ }\NormalTok{dat$dat}
\CommentTok{#smooth depth}
\NormalTok{dat$data$depth <-}\StringTok{ }\NormalTok{stats::}\KeywordTok{filter}\NormalTok{(}\KeywordTok{subset}\NormalTok{(dat$data, }\DataTypeTok{select =} \NormalTok{dat$sensors$depth[}\DecValTok{1}\NormalTok{]),}
                               \DataTypeTok{filter =} \KeywordTok{rep}\NormalTok{(}\DecValTok{1}\NormalTok{,params$smoothing*dat$sensors$sfreq)/}
\StringTok{                                 }\NormalTok{(params$smoothing*dat$sensors$sfreq), }\DataTypeTok{sides =} \DecValTok{2}\NormalTok{, }\DataTypeTok{circular =} \NormalTok{T)}
\CommentTok{#smooth temp reading from depth sensor}
\NormalTok{dat$data$temp.dep <-}\StringTok{ }\NormalTok{stats::}\KeywordTok{filter}\NormalTok{(}\KeywordTok{subset}\NormalTok{(dat$data, }\DataTypeTok{select =} \NormalTok{dat$sensors$depth[}\DecValTok{2}\NormalTok{]),}
                               \DataTypeTok{filter =} \KeywordTok{rep}\NormalTok{(}\DecValTok{1}\NormalTok{,params$smoothing*dat$sensors$sfreq)/}
\StringTok{                                 }\NormalTok{(params$smoothing*dat$sensors$sfreq), }\DataTypeTok{sides =} \DecValTok{2}\NormalTok{, }\DataTypeTok{circular =} \NormalTok{T)}
\CommentTok{#smooth temp imu}
\NormalTok{dat$data$temp.dep <-}\StringTok{ }\NormalTok{stats::}\KeywordTok{filter}\NormalTok{(}\KeywordTok{subset}\NormalTok{(dat$data, }\DataTypeTok{select =} \NormalTok{dat$sensors$temp),}
                               \DataTypeTok{filter =} \KeywordTok{rep}\NormalTok{(}\DecValTok{1}\NormalTok{,params$smoothing*dat$sensors$sfreq)/}
\StringTok{                                 }\NormalTok{(params$smoothing*dat$sensors$sfreq), }\DataTypeTok{sides =} \DecValTok{2}\NormalTok{, }\DataTypeTok{circular =} \NormalTok{T)}

\KeywordTok{plot}\NormalTok{(}\DataTypeTok{x =} \DecValTok{1}\NormalTok{:}\KeywordTok{nrow}\NormalTok{(dat$data), }\DataTypeTok{y =} \NormalTok{dat$data$depth, }\DataTypeTok{type =} \StringTok{"l"}\NormalTok{, }\DataTypeTok{main =} \NormalTok{params$deploy.key)}
\end{Highlighting}
\end{Shaded}

\includegraphics{CATS_preprocess_RMrkD_files/figure-latex/unnamed-chunk-2-1.pdf}

\subsection{Accelerometry}\label{accelerometry}

-currently not making explicit call from UI params list -where to do
pitch/roll (requires accel.. shld mandate accel sensor?)

\begin{verbatim}
## 
## Attaching package: 'gRumble'
\end{verbatim}

\begin{verbatim}
## The following object is masked from 'package:dplyr':
## 
##     collapse
\end{verbatim}

\subsection{Magnetometry}\label{magnetometry}

-based on an email with CFW, 9/5/2018 -not confirmed this works in raw
units -need to double check declination adjustment

\begin{Shaded}
\begin{Highlighting}[]
\NormalTok{if(}\KeywordTok{any}\NormalTok{(}\KeywordTok{str_detect}\NormalTok{(params$sensors, }\StringTok{"^m|M(?=a|A)"}\NormalTok{)))\{}
  \CommentTok{#PUT THIS IN TAG_FXNS?}
  \NormalTok{err <-}\StringTok{ }\NormalTok{function(vars)\{}
    \CommentTok{#seq call is rough/tumble downsample to speed things up}
    \NormalTok{m <-}\StringTok{ }\KeywordTok{as.matrix}\NormalTok{(}\KeywordTok{subset}\NormalTok{(dat$data, }\DataTypeTok{select=}\NormalTok{dat$sensors$mag))}
    \KeywordTok{return}\NormalTok{(}\KeywordTok{sd}\NormalTok{(((m[}\KeywordTok{seq}\NormalTok{(}\DecValTok{1}\NormalTok{, }\KeywordTok{nrow}\NormalTok{(m), dat$sensors$sfreq),}\DecValTok{1}\NormalTok{] -}\StringTok{ }\NormalTok{vars[}\DecValTok{1}\NormalTok{]^}\DecValTok{2}\NormalTok{)+}\StringTok{  }\CommentTok{#mx}
\StringTok{           }\NormalTok{m[}\KeywordTok{seq}\NormalTok{(}\DecValTok{1}\NormalTok{, }\KeywordTok{nrow}\NormalTok{(m), dat$sensors$sfreq),}\DecValTok{2}\NormalTok{] -}\StringTok{ }\NormalTok{vars[}\DecValTok{2}\NormalTok{]^}\DecValTok{2}\NormalTok{)+}\StringTok{       }\CommentTok{#my}
\StringTok{           }\NormalTok{m[}\KeywordTok{seq}\NormalTok{(}\DecValTok{1}\NormalTok{, }\KeywordTok{nrow}\NormalTok{(m), dat$sensors$sfreq),}\DecValTok{3}\NormalTok{] -}\StringTok{ }\NormalTok{vars[}\DecValTok{3}\NormalTok{]^}\DecValTok{2}\NormalTok{))\}     }\CommentTok{#mz}
    
  \NormalTok{out <-}\StringTok{ }\KeywordTok{optim}\NormalTok{(}\KeywordTok{c}\NormalTok{(}\DecValTok{1}\NormalTok{,}\DecValTok{1}\NormalTok{,}\DecValTok{1}\NormalTok{), }\DataTypeTok{fn =} \NormalTok{err)  }\CommentTok{#out$par is the center of the mag data}
  
  \NormalTok{mmms <-}\StringTok{ }\KeywordTok{data.frame}\NormalTok{(}\KeywordTok{Gsep}\NormalTok{(}\KeywordTok{as.matrix}\NormalTok{(}\KeywordTok{subset}\NormalTok{(dat$data, }\DataTypeTok{select =} \NormalTok{dat$sensors$mag)), }
                          \DataTypeTok{filt =} \NormalTok{(}\KeywordTok{rep}\NormalTok{(}\DecValTok{1}\NormalTok{, params$smoothing*dat$sensors$sfreq))/(params$smoothing*dat$sensors$sfreq)))}
  \CommentTok{#downsample}
  \NormalTok{m.ds <-}\StringTok{ }\KeywordTok{data.frame}\NormalTok{(}\KeywordTok{apply}\NormalTok{(mmms, }\DecValTok{2}\NormalTok{, collapse, }
                         \DataTypeTok{freq =} \NormalTok{dat$sensors$sfreq/params$datFreq.desired))}
  
  \CommentTok{#calc heading}
  \NormalTok{h.ds <-}\StringTok{ }\KeywordTok{magHead}\NormalTok{(}\DataTypeTok{PR =} \NormalTok{pr.ds, }\DataTypeTok{magxyz =} \NormalTok{m.ds, }\DataTypeTok{magOff =} \NormalTok{out$par)}
  \CommentTok{#declination at santa cruz ~13.2 deg}
  \NormalTok{declination <-}\StringTok{ }\NormalTok{DescTools::}\KeywordTok{DegToRad}\NormalTok{(}\FloatTok{13.2}\NormalTok{)}
  \KeywordTok{print}\NormalTok{(}\StringTok{"USING DECLINATION ANGLE FOR SANTA CRUZ"}\NormalTok{)}
  \NormalTok{h.ds <-}\StringTok{ }\NormalTok{h.ds +}\StringTok{ }\NormalTok{pi +}\StringTok{ }\NormalTok{declination}
\NormalTok{\}else\{}
  \KeywordTok{warning}\NormalTok{(}\StringTok{"MAG SENSORS WERE NOT INCLUDED IN PARAMS"}\NormalTok{)}
\NormalTok{\}}
\end{Highlighting}
\end{Shaded}

\begin{verbatim}
## [1] "USING DECLINATION ANGLE FOR SANTA CRUZ"
\end{verbatim}

\subsection{GYROMETER}\label{gyrometer}

in dev now

\begin{Shaded}
\begin{Highlighting}[]
\NormalTok{if(}\KeywordTok{any}\NormalTok{(}\KeywordTok{str_detect}\NormalTok{(params$sensors, }\StringTok{"^g|G(?=y|Y)"}\NormalTok{)))\{}
  \CommentTok{#IN DEV}
\NormalTok{\}else\{}
  \KeywordTok{warning}\NormalTok{(}\StringTok{"GYRO SENSORS WERE NOT INCLUDED IN PARAMS"}\NormalTok{)}
\NormalTok{\}}
\end{Highlighting}
\end{Shaded}

\begin{verbatim}
## Warning: GYRO SENSORS WERE NOT INCLUDED IN PARAMS
\end{verbatim}

\subsection{PLOTS}\label{plots}

-ttdr plot needs added functionality for local offsets

\begin{Shaded}
\begin{Highlighting}[]
\KeywordTok{library}\NormalTok{(lubridate)}
\end{Highlighting}
\end{Shaded}

\begin{verbatim}
## 
## Attaching package: 'lubridate'
\end{verbatim}

\begin{verbatim}
## The following objects are masked from 'package:data.table':
## 
##     hour, isoweek, mday, minute, month, quarter, second, wday,
##     week, yday, year
\end{verbatim}

\begin{verbatim}
## The following object is masked from 'package:base':
## 
##     date
\end{verbatim}

\begin{Shaded}
\begin{Highlighting}[]
\NormalTok{x <-}\StringTok{ }\KeywordTok{as.data.frame}\NormalTok{(}\KeywordTok{cbind}\NormalTok{(}\DataTypeTok{dts =} \NormalTok{dat$data$dts[}\KeywordTok{seq}\NormalTok{(}\DecValTok{1}\NormalTok{, }\KeywordTok{nrow}\NormalTok{(dat$data), dat$sensors$sfreq/params$datFreq.desired)],}
                   \KeywordTok{apply}\NormalTok{(}\KeywordTok{subset}\NormalTok{(dat$data, }\DataTypeTok{select =} \KeywordTok{c}\NormalTok{(}\StringTok{"depth"}\NormalTok{, }\StringTok{"temp.dep"}\NormalTok{)), }\DecValTok{2}\NormalTok{, collapse, }
                         \DataTypeTok{freq =} \NormalTok{dat$sensors$sfreq/params$datFreq.desired)))}
\NormalTok{ttdr <-}\StringTok{ }\KeywordTok{ttdr.plot}\NormalTok{(x)}
\NormalTok{ttdr}
\end{Highlighting}
\end{Shaded}

\includegraphics{CATS_preprocess_RMrkD_files/figure-latex/unnamed-chunk-5-1.pdf}


\end{document}
